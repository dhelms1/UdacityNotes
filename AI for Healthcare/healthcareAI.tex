\documentclass[11pt, a4paper]{article}
\usepackage[utf8]{inputenc}
\usepackage[margin=.7in]{geometry}
\usepackage{listings}
\usepackage{setspace}
\usepackage{xcolor}
\usepackage{tcolorbox}
\usepackage{titlesec}
\usepackage{enumitem}
\usepackage{amssymb}
\usepackage{amsmath}
\usepackage{bm}
\usepackage{multicol}
\usepackage{fancybox}
\usepackage{graphicx}
\graphicspath{{./Figures/}}
\usepackage{color}
\usepackage{hyperref}
\hypersetup{
	colorlinks=true,
	linkcolor=blue,
	urlcolor=purple,
}
\titleformat*{\section}{\LARGE\bfseries\filcenter}
\titleformat*{\subsection}{\Large\bfseries}
\titleformat*{\subsubsection}{\large\bfseries}
\definecolor{codegreen}{rgb}{0,0.5,0.3}
\definecolor{codegray}{rgb}{0.5,0.5,0.5}
\definecolor{codered}{rgb}{0.78,0,0}
\definecolor{codepurple}{rgb}{0.58,0,0.68}
\definecolor{backcolour}{rgb}{0.95,0.95,0.92}
\lstdefinestyle{Pystyle}{
	language = Python,
    backgroundcolor=\color{backcolour},   
    commentstyle=\color{gray},
    keywordstyle=\color{black},
    numberstyle=\tiny\color{codepurple},
    stringstyle=\color{codered},
    basicstyle=\ttfamily\footnotesize,
    breakatwhitespace=false,         
    breaklines=true,                 
    captionpos=b,                    
    keepspaces=true,                 
    numbers=left,                    
    numbersep=5pt,                  
    showspaces=false,                
    showstringspaces=false,
    showtabs=false,                  
    tabsize=2,
    morekeywords = {as},
    keywordstyle = \color{codegreen}
}
\lstset{style=Pystyle}
\tcbset{
	colbacktitle=red!50!white, 
	title=Example, 
	coltitle=black, 
	colback=white, 
	fonttitle=\bfseries
}

\begin{document}
	\begin{titlepage}
		\begin{center} \Huge \textbf{Udacity: AI for Healthcare} \end{center}
		\tableofcontents
		\newpage
	\end{titlepage}
	
%%%% PAGE 1 %%%%
	
	\section{Applications to 2D Medical Imaging}
	\subsection{Clinical Foundations of 2D Medical Imaging}
	\begin{itemize}
		\item \textbf{X-ray}: a 2D imaging technique that projects a type of radiation called x-rays down at the body from a single direction to capture a single image.
		\item \textbf{Ultrasound}: a 2D imaging technique that uses high-frequency sound waves to generate images.
		\item \textbf{Computed Tomography (CT)}: a 3D imaging technique that emits x-rays from many different angles around the human body to capture more detail from more different angles.
		\item \textbf{Magnetic Resonance Imaging (MRI)}: a 3D imaging technique that uses strong magnetic fields and radio waves to create images of areas of the body from all different angles.
		\item \textbf{2D imaging}: an imaging technique that pictures are taken from a single angle.
		\item \textbf{3D imaging}: an imaging technique that pictures are taken from different angles to create a volume of images.
	\end{itemize}
	\textbf{X-rays} can show us different absorption levels by the following: bone has high absorption (\textit{bright white}), soft tissue has medium absorption (\textit{gray}), and air has low absorption (\textit{dark}). Chest x-rays for detecting \textbf{pneumonia} will be filled with infiltrates (which are denser with air) and appear as lighter shading in x-rays of lungs. X-rays are stored as single-channel grayscale images and in DICOM format. \vspace*{3mm}\\
	\textbf{PACS} (Picture Archiving and Communication System) are used for storing and accessing images from different areas. \textbf{Diagnostic imaging} occurs when Clinician believes something \textit{may} be wrong with a patient and needs an image to verify (i.e. brain tumors). \textbf{Screening imaging} occurs when nothing actually wrong with patient, but patient is in risk group for disease and need regular imaging to monitor.\vspace*{2mm}\\
	We want to prioritize the most urgent images for radiologists to read based on the probabilities of their being a life threatening finding on these images (between two diseases, we want to read the max first). \vspace*{1mm}
	\begin{lstlisting}
	worklist = pd.read_csv('probabilities.csv')
	worklist['time_to_read'] = np.arange(6, 6*(len(worklist)+1), 6)
	worklist['max_prob'] = worklist.iloc[:, 1:3].max(axis=1)
	worklist.sort_values(by='max_prob', ascending=False, inplace=True)
	worklist['time_to_read_max'] = np.arange(6, 6*(len(worklist)+1), 6)
	worklist['time_delta'] = worklist['time_to_read'] - worklist['time_to_read_max']
	# Check to see how well algorithm had brain bleeds read 30 mins faster
	worklist[(worklist.time_delta > 30) & (worklist.Image_Type == 'head_ct')]
	# Check to see how well algorithm had aortic dissections read 15 mins faster
	worklist[(worklist.time_delta > 15) & (worklist.Image_Type == 'chest_xray')] \end{lstlisting}
	\begin{itemize}
		\item We find the initial time that each image will be read based on a 6 minute read time.
		\item We then order the images based on the max probability between the image types and sort them.
		\item We then find the new read times and find the change in times between that and the original.
		\item We can then check to see how our times improved. 14 head CTs that were read more than 30 minutes faster than their original order. 28 chest x-rays that were read more than 15 minutes faster than their original order.
	\end{itemize}
	\textbf{Sensitivity} measures the proportion of accurately-identified \textit{positive} cases, also called \textit{recall}, and is found by $\frac{TP}{TP + FN}$. \textbf{High Sensitivity Tests} are good for ruling out diseases (if 100\% accuracy then it can find all patients with the disease) and is reliable when the result is negative (rarely misdiagnose people who have disease). \textbf{Specificity} measures the proportion of accurately-identified \textit{negative} cases, and is found by $\frac{TN}{TN+FP}$. \textbf{Highly Specific Tests} are good for ruling in disease (if 100\% accuracy then it can find all patients without the disease) and is reliable when the result is positive (rarely misdiagnose people who don't have disease). \newpage
	
%%%% PAGE 2 %%%%
	
	\noindent To evaluate a segmentation/localization algorithm, we look at the overlap between our algorithm segment ($X$) and radiologist segment ($Y$), and finding the overlap between the two. This can be found by the \textbf{Dice Coefficient}, which is $Dice(X,Y) = \frac{2* |X \cap Y|}{|X| + |Y|}$ 
	
	\subsection{2D Medical Imaging Exploratory Data Analysis}
	\textbf{DICOM} (Digital Imaging and Communications in Medicine) is the standard for the communication and management of medical imaging information and related data. This allows for \textit{interoperability}, which means images from one hospital can be read from any other hospital. A \textbf{series} refers to a single 2D image, and a \textbf{study} is comprised of all image series for the patient. The file contains:
	\begin{itemize}
		\item Header - contains all attributes about an image except the pixel data. This includes information on the patient (name, ID, etc.), the study (date, time, etc.), and the series (number, type, etc.).
		\item Image - contains the pixel data.
	\end{itemize}
	\textbf{Radiologist Report} is not part of the DICOM, but is stored in the PACS and EMR, which tell us details such as location, area of interest, findings, and impression (summary). \vspace*{3mm}\\
	Using the \textbf{pydicom} package, we can read these files and extract the image pixel data from within. We may also want to explore the intensity profiles of these images by plotting histograms. \vspace*{1mm}
	\begin{lstlisting}
	bbox = pd.read_csv('bounding_boxes.csv') # bounding box locations
	dicom1 = pydicom.dcmread('dicom_00013659_019.dcm') # read single dicom series
	dicom1_pdata = dicom1.pixel_array # get pixel data
	plt.imshow(dicom1_pdata, cmap='gray') # see image
	plt.hist(dicom1_pdata.ravel(), bins = 256); # intensity
	
	new_img = dicom1_pdata.copy() # Normalize Image
	new_img = (new_img - np.mean(dicom1_pdata))/np.std(dicom1_pdata)\end{lstlisting} \vspace*{1mm}
	When preparing \textbf{DICOM Headers} to be used in training, we want fast access by pre-extracting all data from DICOM headers into a dataframe. The key attributes are:
	\begin{itemize}
		\item Patient ID - be sure not to include same patient in both train and test set.
		\item Patient's Sex - be sure to have equal male/female proportions in train and test set.
		\item Patient's Age - be sure to have equal age distributions in train and test set.
		\item Study \& Series Instance UID - make sure we don't overtrain on same study or series.
	\end{itemize}
	\begin{lstlisting}
	mydicoms = glob.glob("*.dcm") # gather all dicom files names
	all_data = []
	
	for dicom in mydicoms: # read dicom and save relevant information
		dicom_tmp = pydicom.dcmread(dicom)
		fields = [dicom_tmp.PatientID, dicom_tmp.PatientAge, dicom_tmp.PatientSex, 
		          dicom_tmp.Modality, dicom_tmp.StudyDescription, dicom_tmp.Rows, 
		          dicom_tmp.Columns]
		all_data.append(fields)
		
	mydata = pd.DataFrame(all_data, # create new dataframe with header info
	                      columns = ['PatientID','PatientAge','PatientSex',
	                                 'Modality','Findings','Rows','Columns']) \end{lstlisting}
   	\begin{itemize}
   		\item \textbf{Image artifact}: An object or distortion in an image that reduces its quality
   		\item \textbf{Foreign body}: An object in a medical image that is not biological material from the patient, such as a pacemaker or wire.
   		\item \textbf{Intensity profile}: the distribution of all pixels' intensity values that comprise an image
   	\end{itemize} \newpage

%%%% PAGE 3 %%%%

	\subsection{Classification Models of 2D Medical Images}
	
	
	
	
	
	
	
	
\end{document}

