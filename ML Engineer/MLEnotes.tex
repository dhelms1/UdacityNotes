\documentclass[11pt, a4paper]{article}
\usepackage[utf8]{inputenc}
\usepackage[margin=.7in]{geometry}
\usepackage{listings}
\usepackage{setspace}
\usepackage{xcolor}
\usepackage{titlesec}
\usepackage{enumitem}
\usepackage{amssymb}
\usepackage{amsmath}
\usepackage{bm}
\usepackage{multicol}
\usepackage{graphicx}
\graphicspath{{./Figures/}}
\usepackage{color}
\usepackage{hyperref}
\hypersetup{
	colorlinks=true,
	linkcolor=blue,
	urlcolor=blue,
}
\titleformat*{\section}{\LARGE\bfseries\filcenter}
\titleformat*{\subsection}{\Large\bfseries}
\titleformat*{\subsubsection}{\large\bfseries}
\definecolor{codegreen}{rgb}{0,0.5,0}
\definecolor{codegray}{rgb}{0.5,0.5,0.5}
\definecolor{codered}{rgb}{0.78,0,0}
\definecolor{codepurple}{rgb}{0.58,0,0.68}
\definecolor{backcolour}{rgb}{0.95,0.95,0.92}
\lstdefinestyle{Pythonstyle}{
	language = Python,
    backgroundcolor=\color{backcolour},   
    commentstyle=\color{gray},
    keywordstyle=\color{codegreen},
    numberstyle=\tiny\color{codegray},
    stringstyle=\color{codered},
    basicstyle=\ttfamily\footnotesize,
    breakatwhitespace=false,         
    breaklines=true,                 
    captionpos=b,                    
    keepspaces=true,                 
    numbers=left,                    
    numbersep=5pt,                  
    showspaces=false,                
    showstringspaces=false,
    showtabs=false,                  
    tabsize=2,
    morekeywords = {as},
    keywordstyle = \color{codegreen}
}
\lstset{style=Pythonstyle}

\begin{document}
	\begin{titlepage}
		\begin{center} \Huge \textbf{Udacity: Machine Learning Engineer} \end{center}
		\tableofcontents
		\newpage
	\end{titlepage}
%%%% PAGE 1 %%%%

	\begin{spacing}{1.1}
	\section{Software Engineering Fundamentals}
	\subsection{Software Engineering Practices}
	\textbf{Modular Code}: putting functions into separate files to be imported into workspace. \vspace*{2mm}\\
	\textbf{Refactoring}: restructuring your code to improve its internal structure, without changing its external functionality. This means cleaning and modularizing your program after it is working. \vspace*{2mm}\\
	\textbf{Optimization}: we want to write efficient code, so this can be either fast execution or taking up less space in memory. We also want to \textit{vectorize} our code for speed and amount of coding used.
	\begin{lstlisting}
	for book in recent_books:
		if book in coding_books:
			recent_coding_books.append(book) # 16.63 sec
			
	recent_coding_books = np.intersect1d(recent_books, coding_books) # 0.035 sec
	recent_coding_books = set(recent_books).intersection(coding_books) # 0.0097 sec
	
	for cost in gift_costs:
		if cost < 25:
			total_price += cost * 1.08  # 5.55 sec
			
	total_price = np.sum(gift_costs[gift_costs < 25] * 1.08) # 0.084 sec
	\end{lstlisting} \vspace*{2mm}
	\textbf{Git Branches}: to switch to a branch in a repository you use \textit{git checkout (branchname)}.\\
	To create and switch to a new branch you use \textit{git checkout -b (newbranch)}.\\
	When in main branch, merge another branch by using \textit{git merge -no-ff (branchname)}.\vspace*{2mm}\\
	\textbf{Previous Code}: to see previous commits use \textit{git log}.\\
	Using the commit message number, open the code using a new branch \textit{git checkout (commit\#)}.\vspace*{2mm}\\
	\textbf{Unit Testing}: \href{https://docs.pytest.org/en/latest/getting-started.html}{pytest} is a tool we can use to make sure our function is outputting correctly. We can create a test file starting with \textit{test\_} and we get a . if we pass and an F if we fail.\vspace*{2mm}\\
	\textbf{Test Driven Deployment}: writing tests before you write the code that’s being tested. Your test would fail at first, and you’ll know you’ve finished implementing a task when this test passes.\vspace*{2mm}
	
	\subsection{Object-Oriented Programming}
	
	
	
	
	
	
	
\end{spacing}
\end{document}